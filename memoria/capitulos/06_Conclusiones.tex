\chapter{Conclusiones y vías futuras}
\label{ch:conclusiones}

A partir del estudio realizado en este trabajo se pueden sacar las siguientes conclusiones:
\begin{enumerate}
\item El número de posibles aplicaciones al artículo de Peña y Yohai en \cite{pena16} son amplias y aquí se han tratado sólo unas cuantas, pero sería interesante seguir profundizando en el estudio.
\item Como habíamos previsto, la aplicación a compresión de imágenes no nos ha dado un nuevo método real que se pueda utilizar en compresión de imágenes por el tiempo que se tarda en obtener las componentes principales y la relativa calidad de las imágenes reconstruidas, pero se han obtenido resultados mejores a los esperados inicialmente en cuanto a calidad y ha servido para el propósito principal que tenía: entrar en contacto con el paquete de R que implementa el GDPC y entender mejor el método, así como ver qué cantidad de compresión puede llegar a haber perdiendo en realidad poca información.
\item Se han cumplido de esta forma 3 de los 4 objetivos previstos inicialmente.
\end{enumerate}

Existen dos posibles vías de desarrollo a partir de este momento:
\begin{enumerate}
\item Profundizar en la aplicación de búsqueda de patrones, llegando a dividir una serie en sus distintos elementos e intentando utilizar también el procedimiento GDPC para obtener otros de estos elementos, como la tendencia de la serie.
\item Profundizar en la aplicación de predicción de nuevos valores de un gran conjunto de series a través de la predicción de nuevos valores de las componentes principales dinámicas, haciendo incapié en la utilización de mejores modelos para la predicción de las mismas, utilizando más modelos lineales y no lineales, que no se ha tratado aquí por no ser el tema principal del trabajo.
\end{enumerate}



