%\documentclass[11pt,leqno]{book}
%\usepackage[spanish,activeacute]{babel}
%\usepackage[utf8]{inputenc}
%\usepackage{enumerate}
%\usepackage{float}
%\usepackage{gensymb}
%
%\begin{document}


\chapter{Introducción}
\pagenumbering{arabic}
\setcounter{page}{1}

\section{Contextualización}
La reducción de dimensión se ha convertido, desde hace unos años, en uno de los grandes problemas de la informática, al almacenarse cada vez más información y tener que trabajar con grandes cantidades de datos, lo que conlleva tiempos de ejecución inabarcables, haciéndose muchas veces necesario eliminar partes de la información para poder trabajar con ella. Una de las formas de reducir la dimensión de un conjunto de datos es el análisis de componentes principales. A grandes rasgos, si se dispone de un conjunto de datos de $n$ dimensiones, el análisis de componentes principales busca $m$ nuevos ejes, rotación de los originales, con $m \leq n$, de forma que los datos representados en los nuevos ejes tienen sólo $m$ dimensiones. El número de ejes, es decir, de componentes principales, $m$, se puede elegir en función de la cantidad de información que se está dispuesto a perder, ya que se pueden ir eliminando aquellas componentes que menos influyan en los datos. Intuitivamente, cuanto más relacionados estén los datos entre sí, menos componentes principales serán necesarias. Esto es fácil de ver con el siguiente ejemplo: supongamos que disponemos de 10 datos en $\mathbb{R}^2$, todos alineados. Si obtenemos la primera componente principal, que será la línea por la que pasan los datos, y nos quedamos con la representación de los datos en esta línea, sólo tendremos que almacenar una dimensión, y además lo estamos haciendo sin perder información alguna.\\

Ahora, una de las formas en las que se almacena mucha información es cuando se guarda el valor de una variable a lo largo de periodos de tiempo, por ejemplo, el tiempo en una ciudad a lo largo del año, el número de coches vendidos por meses a lo largo de los últimos 15, 20 o 30 años, el número de defunciones en una ciudad o país desde 1900, o la cantidad de contagios de enfermedades al mes por ciudades. Las posibilidades son infinitas y almacenar y trabajar sobre este tipo de información nos ayuda a predecir qué puede pasar sobre un determinado tema a corto plazo. Además, normalmente este tipo de información no es independiente y depende de varios factores. Por ejemplo, el número de defunciones dependerá del número de contagios de distintas enfermedades en la misma ciudad en ese mes, o el consumo de carburante con la cantidad de coches en circulación. Este tipo de información almacenada se llaman series temporales y es de nuevo interesante reducir la dimensión de las mismas con objeto de trabajar mejor sobre ellas. En esto se centra el paper de Daniel Peña y Víctor Yohai, \cite{pena16}, que aquí se aborda y del que se dan varias posibles aplicaciones.


\section{Descripción del problema abordado}



\section{Técnicas y herramientas utilizadas}


\section{Contenido de la memoria y principales fuentes consultadas}



%\end{document}