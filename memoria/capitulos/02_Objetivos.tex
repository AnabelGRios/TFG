\chapter{Objetivos del trabajo}

Los objetivos previstos inicialmente para este proyecto fueron los siguientes:
\begin{enumerate}
\item Comprender los principios básicos del método \textit{Generalized Dynamic Principal Components (GDPC)} y de los métodos de los que deriva.
\item Aplicar dicho método a compresión de imágenes.
\item Aplicar dicho método a predicción bursátil.
\item Realizar aplicaciones en Brain Computing Interfaces.
\end{enumerate}

El primer y principal objetivo se ha cumplido por completo, entendiendo el análisis de componentes principales dinámicos generalizado. Para ello ha hecho falta buscar y entender también el análisis de componentes principales o PCA, por sus siglas en inglés, ya que éste era una generalización del PCA, y tener una idea básica de lo que son las series temporales, que es el ámbito al que se aplica dicho análisis. Este ha sido el objetivo más difícil y el que mayor tiempo ha llevado, viéndose reflejado en el capítulo \ref{ch:matematicas}.\\

El segundo objetivo también se ha cumplido, aplicando el método a diferentes imágenes para realizar la compresión con éxito. Para ello ha hecho falta la utilización de un software en desarrollo que implementa el método GDPC, y concretamente de dos de sus versiones, la primera de ellas con algunos fallos, y la utilización del software \texttt{R}, como para el resto de aplicaciones que se han llevado a cabo. Este objetivo está ampliamente llevado a cabo en el capítulo \ref{ch:informatica}.\\

El tercer objetivo se ha cumplido aunque no con series bursátiles, pero puede ser extendido a cualquier tipo de serie temporal. También se ve reflejado en el capítulo \ref{ch:informatica}.\\

Por último, el cuarto objetivo no se ha cumplido, pero se han añadido otros objetivos, como la aplicación a detección de patrones en series periódicas.\\


Las asignaturas del doble grado que tienen relación con este trabajo son las siguientes:
\begin{enumerate}
\item Estadísticas descriptiva e introducción a la probabilidad.
\item Geometría I.
\item Geometría II.
\item Aprendizaje Automático.
\item Estadística Computacional.
\item Fundamentos de Programación.
\item Análisis I.
\item Cálculo I.
\item Cálculo II.
\end{enumerate}